\chapter{Role of convergence in the theory of metric spaces}

\section{Sequences in metric spaces and their convergence}
{\color{blue}

We will include selected definitions and theorems, for proofs see \cite{cech66}.
\begin{define}\label{def:mp}
Let $X$ be a set. Let $\varrho: X\times X \to [0,\infty)$ be a mapping satisfying the following conditions:
\begin{enumerate}
	\item[(M1)] For $x,y\in X: \varrho(x,y)=0 \Leftrightarrow x = y$.
	\item[(M2)] For $x,y,z\in X: \varrho(x,z) \le \varrho(x,y) + \varrho(z,y)$.
\end{enumerate}
We say that $\varrho$ is a \emph{metric on $X$} and $(X,\varrho)$ is a \emph{metric space}.
\end{define}
Condition \emph{(M1)} is called \emph{Identity of indiscernibles} and \emph{(M2)} is called \emph{Triangle inequality}. It is an easy exercise to show that following holds:
\begin{enumerate}
	\item[(M3)] For $x,y\in X: \varrho(x,y)=\varrho(y,x)$.
\end{enumerate}
Sometimes \emph{(M2)} is writen with $\varrho(y,z)$ instead of $\varrho(z,y)$ but then \emph{(M3)} must be added to the conditions.
}
\section{Some known properties of sequences in metric spaces}

\section{Historical notes}

