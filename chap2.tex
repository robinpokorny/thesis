\chapter{Sequential spaces and their properties} 

\section{Family of convergent sequences on a set}

\begin{define}\label{def:fcs}
Let $X$ be a non-empty set. A \emph{family of convergent sequences on $X$} \tbd(jiný název) (denoted by $\Xi_X$) is a set of pairs $(\{x_n\}^\infty_{n=1},x)$ consisting of a sequence $\{x_n\}^\infty_{n=1}\subseteq X$ and a point $x\in X$ satisfying the following conditions:
\begin{enumerate}
	\item[(C1)] If $(\{x_n\}^\infty_{n=1},x)\in\Xi_X$, $(\{y_n\}^\infty_{n=1},y)\in\Xi_X$ and $\{y_n\}^\infty_{n=1}$ is a subsequence of $\{x_n\}^\infty_{n=1}$ then $x=y$.
	\item[(C2)] If $x\in X$ then $(\{x\}^\infty_{n=1},x)\in\Xi_X$.
	\item[(C3)] If  $(\{x_n\}^\infty_{n=1},x)\notin\Xi_X$ then there exists a subsequence $\{x_{n_k}\}^\infty_{k=1}$ such that for none of its subsequence $\{x_{n_{k_i}}\}^\infty_{i=1}$ is $(\{x_{n_{k_i}}\}^\infty_{i=1},x)\in\Xi_X$.
\end{enumerate}
\end{define}

In the next we will write $\Xi$ instead of $\Xi_X$ wherever the set $X$ is clear from the context. Unless stated otherwise $X$ will denote a non-empty set. When $(\{x_n\},x)\in\Xi$ we say that $\{x_n\}$ is  a \emph{convergent sequence}, $\{x_n\}$ \emph{converges} to $x$ or that $x$ is a limit of  $\{x_n\}$.

The properties in definition \ref{def:fcs} can be restated as follows:
\begin{enumerate}
	\item[(C1')] A subsequence of a convergent sequence is convergent and converges to the same point.
	\item[(C2')] A sequence consisting of one point (constant sequence) converges to that point.
	\item[(C3')] A non-convengent sequence (that is not converging to point $x$) contains a subsequence such that any of its subsequences either converges to different point (not to $x$) or does not converge at all.
\end{enumerate}
The first two properties are natural and follows from basic knowledge of convergence in metric spaces. To understand the condition \emph{(C3)} we need to remind that a sequence $\{x_n\}$ which does not converge to $x$ might include a subsequence $\{x_{n_k}\}$ which converges to $x$.

\begin{define}\label{def:gen}
Let $(X,\rho)$ be a metric space. We say $\Xi$ is {\emph generated by} $X$ if 
\[
	\Xi=\{(\{x_n\},x): x\in X, \forall n\in \mathbb{N}\ x_n\in X, \lim_{n \to \infty} \rho(x_n,x)=0\}.
\]
\end{define}

The next proposition shows that every sequence has at most one limit.

\begin{theorem} \label{th:onelim}
Let $(\{x_n\},x)\in\Xi$. Then $\forall y\in X\setminus\{x\}: (\{x_n\},y)\notin\Xi$.
\end{theorem} 
\begin{proof} 
	A sequence is a subsequence of itself. So when $(\{x_n\},x)\in\Xi$ and $(\{x_n\},y)\in\Xi$ than from \emph{(C1)} we have $x=y$.
\end{proof}

\begin{define}\label{def:ekv}
Let $(\{x_n\},x)\in\Xi$ and $(\{y_n\},y)\in\Xi$, we say that sequences $\{x_n\}$ and $\{y_n\}$ are \emph{equivalent} if $\exists\, n_0, n_1 \in \mathbb{N}\ \forall n \in \mathbb{N}: x_{n_0+n}=y_{n_1+n}$. We write $\{x_n\} \sim \{y_n\}$
\end{define}

Equivalence is well defined. When $\{x_n\} \sim \{y_n\}$ clearly $\{y_n\} \sim \{x_n\}$.
If $\{x_n\} = \{y_n\}$ which we can rewrite as $x_n = y_n\ \forall n \in \mathbb{N}$ and we set $n_0 = n_1 = 1$. Let $\{x_n\}, \{y_n\}$ and $\{z_n\}$ be sequences such that

\[
	\exists\, n_0, n_1, m_0, m_1 \in \mathbb{N}\ \forall n \in \mathbb{N}: x_{n_0+n}=y_{n_1+n},\ y_{m_0+n}=z_{m_1+n}.
\]
That is $\{x_n\} \sim \{y_n\}$ and $\{y_n\} \sim \{z_n\}$.  To obtaint $\{x_n\} \sim \{z_n\}$ we set $k_0=n_0+m_0$ and $k_1=m_1+n_1$ and we have
\[
	x_{k_0+n}=x_{n_0+m_0+n}=y_{n_1+m_0+n}=z_{m_1+n_1+n}=z_{k_1+n}.
\]


\begin{theorem} \label{th:eqseq}
Let $\Xi$ be a family of convergent sequences, $(\{x_n\},x), (\{y_n\},y)\in\Xi$ and $\{y_n\} \sim \{x_n\}$. Then $x=y$.
\end{theorem} 
\begin{proof}
We use \emph{(C1)} twice.
\end{proof}

When we have $(\{x_n\},x)\in\Xi$ and an arbitrary point $x\in X$, new sequences can be defined:
\[
	y_1=x,\ y_{n+1}=x_n,\qquad z_n=x_{n+1},\qquad\forall n\in \mathbb{N}.
\]
Both of these new sequences are equivalent with $\{x_n\}$ and therefore by Theorem~\ref{th:eqseq} they have the same limit - if present in $\Xi$. The definition of family of convergent sequences does not say that they do converge, it would however be odd and unnatural when removing (or adding) one member (or any finite number of members) affected the convergence of the sequence. One way to solve this would be to define something like complete family of convergent sequences i.e. when $(\{x_n\},x)\in\Xi$ and $\{y_n\} \sim \{x_n\}$ then $(\{y_n\},x)\in\Xi$. This would nevertheless require us to write unnecessary statements like "up to equivalence". When defined so complete family of convergent sequences would not add any new information about properties we study. To simplify the technique and the language we will develop a quotient space. That means if there is convergent sequence with limit $x$ all its equivalent sequences with the same limit $x$ forms one point in this space. 

\begin{define}\label{def:quosp}
Let $\Xi$ be a family of convergent sequences. Let us denote by $\Xi/\!\sim$ the \emph{set of all equivalence classes of $\sim$}:
\begin{eqnarray}
	\Xi/\!\sim & := & \bigg\{\Big[(\{x_n\},x)\Big] : (\{x_n\},x)\in\Xi \bigg\} = \nonumber\\
	& {}= & \bigg\{\Big\{(\{y_n\},x), \{y_n\} \sim \{x_n\}\Big\} : (\{x_n\},x)\in\Xi \bigg\}.\nonumber
\end{eqnarray}
\end{define}

Note that we do not demand $ \{y_n\}$ to be convergent, but we handle it as it is. In the following we will write $[\{x_n\},x]$ or instead of $[(\{x_n\},x)\Big]$ and $\Xi$ instead of $\Xi/\!\sim$.


\begin{define}\label{def:xi0}
Let $\Xi$ be a family of convergent sequences. We define
\[
	\Xi_0:=\bigg\{[\{x\},x] : x\in X\bigg\}.
\]
\end{define}
The definitions says that $\Xi_0$ consists of convergent sequences which are constant up to a finite number of members. From \emph{(C2)} if follows $\Xi_0\subseteq \Xi$ so $\Xi_0$ is minimal family of convergent sequences on the set. It is wherefore interesting to see for which metric spaces $\Xi=\Xi_0$ holds.

\begin{theorem} \label{th:xieqxi0}
Let $\Xi$ be generated by metric space $(X,\rho)$ then following are equivalent:
\begin{enumerate}[(i)]
	\item $\Xi=\Xi_0$
	\item $\forall x\in X,\ \exists \varepsilon_x>0,\ \forall y\in X, x\neq y: \rho(x,y)>\varepsilon_x$.
	\item $(X,\rho)$ is topologically discrete.
\end{enumerate}
\end{theorem} 

\begin{proof}
	The space $X$ being discrete, we know there exists $\varepsilon >0, \forall x,y\in X, y\neq x: \rho(x,y)>\varepsilon$.
	When $\{x_n\}$ converges to $x$ it has to consist only $x$ from some index in order to meet the definition of a limit in a metric space.
	
	Let $\Xi=\Xi_0$ and let $X$ be not discrete, that is $\forall \varepsilon>0\ \exists x\in X\ \exists y\in X: \rho(x,y)\leq \varepsilon$. Assume that $X$ is complete
\end{proof}

\begin{example}
	\tbd\ Difference beetween metric dicrete and topological discrete space !!! $X=\{\frac{1}{n}, n\in N\}$, !!! $[0,1)$, !!! $\{(n,0), n\in N\}\cup\{(n,\frac{1}{n}), n\in N\}$
\end{example}


\

\

!Ekvivalence a úplnost


\section{Open and closed sets in sequential spaces and related topological properties}

We define a preclosure operator:
\begin{enumerate}
	\item[(PO)] A point lies in the closure of a set iff there is a sequence in the set converging to the point.
\end{enumerate}

\section{Compactness}

\section{Complete spaces}

\section{Bounded and totally bounded spaces}

\begin{theorem} \label{th:xieqxi0}
When $(X,\varrho)$ is an unbounded metric space then there exists a metric $\sigma$ on $X$ that is equivalent with $\varrho$ and $(X,\sigma)$ is bounded.
\end{theorem}
\begin{proof}
For $x,y\in X$ we define
\[
	\sigma_1(x,y):=\frac{\varrho(x,y)}{1+\varrho(x,y)},\qquad\sigma_2(x,y):=\min\{\varrho(x,y), 1\}.
\]
See \cite[p.~22]{copson88} for $\sigma_1$ and \cite[p.~250]{engelking89} for $\sigma_2$ for proofs that they satisfy the conditions for a metric. Then evidently both $\sigma_1, \sigma_2$ are equivalent with $\varrho$ and bounded.
\end{proof}

The previous theorem shows that we can not \tbd


\section{Connected spaces}

\section{Separable spaces}
