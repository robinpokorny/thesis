\chapter{Sequential spaces and their metric properties} 

\section{Family of convergent sequences on a set}

\begin{define}\label{def:fcs}
Let $X$ be a set. A {\sl family of convergent sequences on $X$} (denoted by $\Xi_X$) is a set of pairs $(\{x_n\}^\infty_{n=1},x)$ consisting of a sequence $\{x_n\}^\infty_{n=1}\subseteq X$ and a point $x\in X$ satisfying the following conditions:
\begin{enumerate}
	\item[(C1)] If $(\{x_n\}^\infty_{n=1},x)\in\Xi_X$, $(\{y_n\}^\infty_{n=1},y)\in\Xi_X$ and $\{y_n\}^\infty_{n=1}$ is a subsequence of $\{x_n\}^\infty_{n=1}$ then $x=y$.
	\item[(C2)] If $x\in X$ then $(\{x\}^\infty_{n=1},x)\in\Xi_X$.
	\item[(C3)] If  $(\{x_n\}^\infty_{n=1},x)\notin\Xi_X$ then there exists a subsequence $\{x_{n_k}\}^\infty_{k=1}$ such that for none of its subsequence $\{x_{n_{k_i}}\}^\infty_{i=1}$ is $(\{x_{n_{k_i}}\}^\infty_{i=1},x)\in\Xi_X$.
\end{enumerate}
\end{define}

In the next we will write $\Xi$ instead of $\Xi_X$ wherever the set $X$ is clear from the context; we will also sometimes leave the indexes of a sequence and thus write only $\{x_n\}$. Unless stated otherwise $X$ will denote a non-empty set. When $(\{x_n\},x)\in\Xi$ we say that $\{x_n\}$ is a {\sl convergent sequence}, $\{x_n\}$ \sl{converges} to $x$ or that $x$ is a limit of  $\{x_n\}$.

The properties in definition \ref{def:fcs} can be restated as follows:
\begin{enumerate}
	\item[(C1')] A subsequence of a convergent sequence is convergent and converges to the same point.
	\item[(C2')] A sequence consisting of one point (constant sequence) converges to that point.
	\item[(C3')] A non-convengent sequence \tbd
\end{enumerate}
The first two properties are natural and follows from basic knowledge of convergence in metric spaces. To understand the condition (C3) we need to remind that a sequence $\{x_n\}$ which does not converge to $x$ might include a subsequence $\{x_{n_k}\}$ which converges to $x$. The next proposition shows that every sequence has at most one limit.

\begin{theorem} \label{th:onelim}
Let $(\{x_n\},x)\in\Xi$. Then $\forall y\in X\setminus\{x\}: (\{x_n\},y)\notin\Xi$.
\end{theorem} 

\begin{proof} 
\tbd
\end{proof}

\begin{define}\label{def:ekv}
Let $(\{x_n\},x)\in\Xi$ and $(\{y_n\},y)\in\Xi$, we say that $\{x_n\}$ and $\{y_n\}$ are {\sl equivalent} if $\exists\, n_0, n_1 \in \mathbb{N}\ \forall n \in \mathbb{N}: x_{n_0+n}=y_{n_1+n}$. We write $\{x_n\} \sim \{y_n\}$
\end{define}

\begin{define}\label{def:xi0}
Let $\Xi$ be a family of convergent sequences. We define
\[
	\Xi_0:=\{(\{x_n\},x)\in\Xi : \exists y\in X\ \{x_n\} \sim \{y\}\}.
\]
\end{define}
Evidently $\Xi_0\subseteq \Xi$. The definitions says that $\Xi_0$ consists of convergent sequences which are constant up to a finite number of members. 

\begin{define}\label{def:gen}
Let $(X,\rho)$ be a metric space. We say $\Xi$ is {\sl generated by} $X$ if 
\[
	\Xi=\{(\{x_n\},x): x\in X, \forall n\in \mathbb{N}\ x_n\in X, \lim_{n \to \infty} \rho(x_n,x)=0\}.
\]
\end{define}

\begin{theorem} \label{th:xieqxi0}
Let $\Xi$ be generated by metric space $(X,\rho)$ than following are equivalent:
\begin{enumerate}[(i)]
	\item $\Xi=\Xi_0$
	\item $X$ is finite or $(X,\rho)$ is discrete.
\end{enumerate}
\end{theorem} 

\begin{proof} 
\tbd
\end{proof}



\

\

!Ekvivalence a úplnost

!Preclosure operator


\section{Open and closed sets in sequential spaces and related topological properties}

\section{Compactness}

\section{Complete spaces}

\section{Bounded and totally bounded spaces}

\section{Connected spaces}

\section{Separable spaces}
