\chapter{Sequential and uniformly sequential spaces}

\section{Family of convergent sequences on a set}

To study the role of convergence in metric spaces we take some properties of sequences which hold in metric spaces and create new space using these properties as conditions. It turns out that the following conditions \emph{(C1)} - \emph{(C4)} introduce a rich class of spaces yet relatively simple to explore. 

\begin{define}\label{def:fcs}
Let $X$ be a non-empty set. A \emph{\fcs\ on $X$} (denoted by $\Xi_X$) is a set of pairs $(\{x_n\},x)$ consisting of a sequence $\{x_n\}\subseteq X$ and a point $x\in X$ satisfying the following conditions:
\begin{enumerate}
	\item[(C1)] If $(\{x_n\},x)\in\Xi_X$ and $\{y_n\}$ is a subsequence of $\{x_n\}$ then $(\{y_n\},x)\in\Xi_X$.
	\item[(C2)] If $(\{x_n\},x)\in\Xi_X$, $(\{x_n\},y)\in\Xi_X$ then $x=y$.
	\item[(C3)] If  $(\{x_n\},x)\notin\Xi_X$ then there exists a subsequence $\{x_{n_k}\}^\infty_{k=1}$ such that for none of its subsequence $\{x_{n_{k_i}}\}$ is $(\{x_{n_{k_i}}\},x)\in\Xi_X$.
	\item[(C4)] If $x\in X$ then $(\{x\},x)\in\Xi_X$.
\end{enumerate}
We say $(X,\Xi_X)$ is a \emph{sequential space}.
\end{define}

In the next we will write $\Xi$ instead of $\Xi_X$ wherever the set $X$ is clear from the context. Unless stated otherwise $X$ will denote a non-empty set. When $(\{x_n\},x)\in\Xi$ we say that $\{x_n\}$ is  a \emph{convergent sequence}, $\{x_n\}$ \emph{converges} to $x$ or that $x$ is a limit of  $\{x_n\}$.

The properties in Definition \ref{def:fcs} can be restated as follows:
\begin{enumerate}
	\item[(C1')] A subsequence of a convergent sequence is convergent and converges to the same point.
	\item[(C2')] Every sequence has at most one limit.
	\item[(C3')] A sequence that is not converging to a point $x$ contains a subsequence such that any of its subsequences either converges to a different point (not to $x$) or does not converge at all.
	\item[(C4')] A constant sequence converges to the common point.
\end{enumerate}
The first two and the fourth properties are natural and follow from basic knowledge of convergence in metric spaces. To understand the condition \emph{(C3)} we need to remind that a sequence $\{x_n\}$ which does not converge to $x$ might include a subsequence $\{x_{n_k}\}$ which converges to $x$.

Our motivation to define \fcss\ was to generalise convergence in metric spaces. The following shows the relation between them.

\begin{define}\label{def:gen}
Let $(X,\varrho)$ be a metric space. We say $\Xi$ is \emph{generated by} $X$ if 
\[
	\Xi=\{(\{x_n\},x): x\in X, \forall n\in \mathbb{N}, \ x_n\in X, \lim_{n \to \infty} \varrho(x_n,x)=0\}.
\]
\end{define}

A \fcs\ generated by a metric space is well defined. We know that $\lim_{n\to\infty}\varrho(x_n,x)=0$ implies $\lim_{n\to\infty}\varrho(x_{n_k},x)=0$ for every subsequence $\{x_{n_k}\}$ of $\{x_n\}$. If $x_n\to x$ and $x_n\to y$ for some $x,y\in X$ we have
\[
	0\le\varrho(x,y)\le\varrho(x_n,x)+\varrho(x_n,y)\to 0
\]
as $n\to\infty$, and so $\varrho(x,y)=0$ which by \emph{(M1)} means $x=y$. Property \emph{(C3)} follows from Theorem \ref{th:mpseq}. And last, obviously $\{x\}\to x$. 

In metric spaces adding a finite number of members to a sequence as well as removing finite number of members from it does not affect the convergence of the sequence. The same holds for \fcss.

\begin{theorem} \label{th:eqseq}
Let $\Xi$ be a \fcs, $(\{x_n\},x)\in\Xi$ and $a\in X$. Let $\{y_n\}$ and $\{z_n\}$ be sequences defined as follows:
\begin{enumerate}[(i)]
	\item $y_1=a$, $y_{n+1}=x_n\quad \forall n\in \mathbb{N}$,
	\item $z_n=x_{n+1}\quad \forall n\in \mathbb{N}$.
\end{enumerate}
Then $(\{y_n\},x), (\{z_n\},x)\in\Xi$.
\end{theorem} 
\begin{proof}
Let for contradiction suppose $(\{y_n\},x)\notin\Xi$. Then using \emph{(C3)} we can find a subsequence $\{y_{n_k}\}$ such that for every its subsequence $(\{y_{n_{k_j}}\},x)\notin\Xi$. We can assume $\{y_{n_{k_j}}\}$ does not include the first member of $\{y_n\}$. Then $\{y_{n_{k_j}}\}$ is a subsequence of $\{x_n\}$ and by \emph{(C1)} we have $(\{y_{n_{k_j}}\},x)\in\Xi$ which is a contradiction.

Since $\{z_n\}$ is a subsequence of $\{x_n\}$ the conclusion follows from \emph{(C1)}.
\end{proof}

\begin{theorem} \label{th:intxi}
An intersection of \fcss\ on a set $X$ is a \fcs\ on $X$.
\end{theorem} 
\begin{proof}
Let $\Xi_\alpha$ be a \fcs\ on $X$ for every $\alpha\in A$, where $A$ is an arbitrary non-empty set. Let $(\{x_n\},x)\in \bigcap_{\alpha\in A}\Xi_\alpha$ and $\{y_n\}$ be a subsequence of $\{x_n\}$. Then for every $\alpha\in A$ we have $(\{x_n\},x)\in \Xi_\alpha$ so by \emph{(C1)} we have $(\{y_n\},x)\in \Xi_\alpha$. That is, $(\{y_n\},x)\in \bigcap_{\alpha\in A}\Xi_\alpha$.

Also for every $\alpha\in A$ and every $x\in X$ we have $(\{x\},x)\in \Xi_\alpha$ so $(\{x\},x)\in \bigcap_{\alpha\in A}\Xi_\alpha$.

Let $(\{x_n\},x)\not\in \bigcap_{\alpha\in A}\Xi_\alpha$ then there exists $\alpha_0\in A$ for which $(\{x_n\},x)\not\in \Xi_{\alpha_0}$. We find the subsequence $\{x_{n_k}\}$ from \emph{(C3)} for $\Xi_{\alpha_0}$ and let $\{x_{n_{k_i}}\}$ be its subsequence. Then $(\{x_{n_{k_i}}\},x)\not\in\Xi_{\alpha_0}$ therefore $(\{x_{n_{k_i}}\},x)\not\in \bigcap_{\alpha\in A}\Xi_\alpha$.

Let $(\{x_n\},x)\in \bigcap_{\alpha\in A}\Xi_\alpha$ and $y\in X,\ x\ne y$. Then for some $\alpha\in A$ we have $(\{x_n\},x)\in \Xi_\alpha$ and from \emph{(C2)} for $\Xi_\alpha$ we have $(\{x_n\},y)\not\in \Xi_\alpha$. Therefore $(\{x_n\},y)\not\in \bigcap_{\alpha\in A}\Xi_\alpha$.
\end{proof}

\begin{define}\label{def:xi0}
Let $X$ be a non-empty set. Symbol $\Xi_0$ will denote the smallest (in the meaning of inclusion) \fcs\ on X.
%for which the following holds: \[\Xi_0\supset\bigg\{(\{x\},x) : x\in X\bigg\}.\]
\end{define}

%The existence of such \fcs\ follows from the Zorn's lemma and Theorem \ref{th:intxi}.
The definition says that $\Xi_0$ consists of convergent sequences which are constant up to a finite number of members.

Since $\Xi_0$ is the smallest \fcs\ on the set it is wherefore interesting to see for which metric spaces $\Xi=\Xi_0$ holds. It is easy to observe that $\Xi$ generated by a metrically discrete space (in Example \ref{ex:msp}) has his property.


At first sight one might think that  $\Xi=\Xi_0$ implies the space $(X,\varrho)$ to be metrically discrete. The following shows a counterexample constructing a space that is not metrically discrete.
\begin{example}
	Suppose $X=\{\frac{1}{n}, n\in \mathbb{N}\}$, then $X\subset \mathbb{R}$ and with induced metric it forms a subspace of $\mathbb{R}$. As the sequence $\{\frac{1}{n}\}$ has a limit in $\mathbb{R}$ which is not included in $X$, every convergent sequence in $X$ has to be constant from some index. Therefore $\Xi=\Xi_0$.
\end{example}

Another idea we had was: If $\Xi=\Xi_0$ then either $X$ is metrically discrete or $X$ is not complete. The following example prooves this statement to be false.

\begin{example}
	Suppose $X=\{(n,0), n\in \mathbb{N}\}\cup\{(n,\frac{1}{n}), n\in \mathbb{N}\}$, then $X\subset \mathbb{R}^2$. Intuitive image is the set of natural numbers as a subset of $x$ axis united with the graph of $\{\frac{1}{n}\}$ in the plane. This space (with induced metric) is complete and every convergent sequence is constant from some index.
\end{example}


\begin{theorem} \label{th:genxi0}
Let $\Xi$ be generated by a metric space $(X,\varrho)$ then the following are equivalent:
\begin{enumerate}[(i)]
	\item $\Xi=\Xi_0$;
	\item $\forall x\in X,\ \exists \varepsilon_x>0,\ \forall y\in X, x\neq y: \varrho(x,y)>\varepsilon_x$. (We say $(X,\varrho)$ is \emph{topologically discrete}.)
\end{enumerate}
\end{theorem} 

\begin{proof}
Let $X$ be a topologically discrete metric space, $\Xi$ generated by $X$ and for contradiction suppose there exists a sequence $\{x_n\}$ and $x\in X$ such that $(\{x_n\},x)\in\Xi\setminus\Xi_0$. Then from (ii) we find $\varepsilon_x$. Because $\lim_{n\to\infty}\varrho(x_n,x)=0$ there is $n_0\in\mathbb{N}$ for which $\varrho(x_{n_0},x)\le\frac{1}{2}\varepsilon_x$ and $x_{n_0}\ne x$ (since $(\{x_n\},x)\notin\Xi_0$). This is a contradiction with $X$ being topologically discrete.

Let $\Xi=\Xi_0$ and for contradiction suppose that for some $x\in X$ we have
\[
	\forall \varepsilon>0,\ \exists y\in X, x\ne y: \varrho(x,y)<\varepsilon.
\]
For $n\in \mathbb{N}$ we set $\varepsilon_n:=\frac{1}{n}$ and find $y_n\ne x$ such that $\varrho(x,y_n)<\varepsilon_n$. Then $\{y_n\}\to x$ which is a contradiction since $\{y_n\}$ is not constant.
\end{proof}

\section{Uniformly sequential structure on a set}

%\begin{define}{Properties needed later \tbd} \label{def:cofin}
%\begin{enumerate}[(i)]
%	\item $A'$ is cofinal in $A$.
%\end{enumerate}
%\end{define}

% \{x_n\} 

\begin{define} \label{def:unistr}
Let $X$ be a non-empty set.  A \emph{\uss\ on $X$} is an equivalence $\sim$ on sequences in $X$ satisfying the following conditions:
\begin{enumerate}
	\item[(U1)] If $\{x_n\} \sim \{y_n\}$ and $\{x_{n_k}\}, \{y_{n_k}\}$ are subsequences then $\{x_{n_k}\} \sim \{y_{n_k}\}$.
	\item[(U2)] If $x_n=x, y_n=y\ \forall n\in \mathbb{N}$ for some $x,y\in X$ and $\{x_n\} \sim \{y_n\}$ then $x=y$.
	\item[(U3)] If $\{x_n\} \not\sim \{y_n\}$ then there exist subsequences $\{x_{n_k}\}, \{y_{n_k}\}$ such that for none of their subsequences $\{x_{n_{k_i}}\}, \{y_{n_{k_i}}\}$ is $\{x_{n_{k_i}}\}\sim \{y_{n_{k_i}}\}$.
	\item[(U4)] If $x\in X$ then $\{x\} \sim \{y\}$. (Reflexivity) 
\end{enumerate}
We say $(X,\sim)$ is a \emph{uniformly sequential space}. If $\{x_n\} \sim\{y_n\}$ we say  \emph{$\{x_n\} $ is adjacent to $\{y_n\}$}.
\end{define}

\begin{theorem} \label{th:unitoseq}
	Let $X$ have a \uss. Then
\[
	\Xi_\sim:=\bigg\{(\{x_n\},x): \{x_n\}\sim \{x\}\bigg\}
\]
is a \fcs\ on $X$.
\end{theorem}
\begin{proof}
Let $\{x_n\}\sim \{x\}$ and $\{x_{n_k}\}$ be subsequence of $\{x_n\}$. We know every subsquence of a constant sequence is again a constant sequence. So from \emph{(U1)} we have $\{x_{n_k}\}\sim \{x\}$.

Let $x,y\in X$. Let $\{x_n\} \sim \{x\}$ and $\{x_n\} \sim \{y\}$. Then from transitivity of $\sim$ we obtain $\{x\} \sim \{y\}$ and \emph{(U2)} gives us $x=y$.

Let $\{x_n\} \not\sim \{x\}$. Then \emph{(C3)} follows directly from \emph{(U3)}.

Because of \emph{(U4)} we have $(\{x\},x)\in\Xi_\sim$.
\end{proof}

\begin{theorem}\label{th:adjprop}
Adjancent sequences in a metric space $(X,\varrho)$ satisfy conditions \emph{(U1)-(U4)}.
\end{theorem}
\begin{proof}
From Theorem \ref{th:adjquiv} we know  $\sim_\varrho$ is an equivalence.

Let $x_n=x, y_n=y\ \forall n\in \mathbb{N}$ for some $x,y\in X$ and $\{x_n\} \sim_\varrho \{y_n\}$. If $x\ne y$ then from \emph{(M1)} we have $\varrho(x,y)>0$ so $x=y$.

Let $\{x_n\} \sim_\varrho \{y_n\}$ and $\{x_{n_k}\}, \{y_{n_k}\}$ be subsequences. Then clearly $\{x_{n_k}\}\sim_\varrho \{y_{n_k}\}$ .

Let $\{x_n\} \not\sim_\varrho \{y_n\}$ then there exists $\varepsilon>0$ such that for infinitely many $n \in \mathbb{N}$ we have $\varrho(x_n,y_n)\ge\varepsilon$, these introduce a subsequence. It is easy to observe it has desired properties.
\end{proof}

We are not able to characterise a Cauchy sequence in terms of simple convergence. It turns out that this problem is the main motivation to develop uniformly sequential spaces in which we can define this property.

\begin{define}\label{def:ucauch}
Let $X$ have a \uss. We say a sequence $\{x_n\}$ in $X$ is \emph{Cauchy} if it is adjacent to all its subsequences.
\end{define}

As we would expect every convergent sequence is Cauchy.

\begin{theorem}\label{th:cauchconvg}
Let $X$ have a \uss. If $(\{x_n\}, x)\in\Xi_\sim$ then $\{x_n\}$ is Cauchy.
\end{theorem}
\begin{proof}
Let $\{x_{n_k}\}$ be a subsequence of $\{x_n\}$. Then from \emph{(C1)} we have $(\{x_{n_k}\}, x)\in\Xi_\sim$. That is $\{x_n\} \sim \{x\}$ and $\{x_{n_k}\} \sim \{x\}$ and the transitivity of $\sim$ gives us $\{x_{n_k}\} \sim \{x_n\}$.
\end{proof}

The relation of a Cauchy sequence in a uniformly sequential space and a Cauchy sequence in a metric space is shown later (Theorem \ref{th:scomplcompl}).

%Poznámka: (Více na Császar: GT, 1978 - str. 153 for totally bounded uniform spaces, str. 139 for uniform spaces)




















