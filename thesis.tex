%%% Hlavní soubor. Zde se definují základní parametry a odkazuje se na ostatní části. %%%

%% Verze pro jednostranný tisk:
% Okraje: levý 40mm, pravý 25mm, horní a dolní 25mm
% (ale pozor, LaTeX si sám přidává 1in)
\documentclass[12pt,a4paper]{report}
\setlength\textwidth{145mm}
\setlength\textheight{247mm}
\setlength\oddsidemargin{15mm}
\setlength\evensidemargin{15mm}
\setlength\topmargin{0mm}
\setlength\headsep{0mm}
\setlength\headheight{0mm}
% \openright zařídí, aby následující text začínal na pravé straně knihy
\let\openright=\clearpage

%% Pokud tiskneme oboustranně:
% \documentclass[12pt,a4paper,twoside,openright]{report}
% \setlength\textwidth{145mm}
% \setlength\textheight{247mm}
% \setlength\oddsidemargin{15mm}
% \setlength\evensidemargin{0mm}
% \setlength\topmargin{0mm}
% \setlength\headsep{0mm}
% \setlength\headheight{0mm}
% \let\openright=\cleardoublepage

\usepackage[british]{babel}
%% Pokud používáte csLaTeX (doporučeno):
%\usepackage{czech}
%% Pokud nikoliv:
%\usepackage[czech]{babel}
%\usepackage[T1]{fontenc}

%% Použité kódování znaků: obvykle latin2, cp1250 nebo utf8:
\usepackage[utf8]{inputenc}

%% Ostatní balíčky
\usepackage{graphicx}
\usepackage{amsthm}
\usepackage{amssymb}
\usepackage{enumerate}
\usepackage{color}

%% Balíček hyperref, kterým jdou vyrábět klikací odkazy v PDF,
%% ale hlavně ho používáme k uložení metadat do PDF (včetně obsahu).
%% POZOR, nezapomeňte vyplnit jméno práce a autora.
\usepackage[pdftex,unicode]{hyperref}   % Musí být za všemi ostatními balíčky
\hypersetup{
	pdftitle={Properties of metric spaces by means of convergence},
	pdfauthor={Robin Pokorný}
%	, pdfkeywords={keyword1} {key2} {key3}
}

%%% Drobné úpravy stylu

% Tato makra přesvědčují mírně ošklivým trikem LaTeX, aby hlavičky kapitol
% sázel příčetněji a nevynechával nad nimi spoustu místa. Směle ignorujte.
\makeatletter
\def\@makechapterhead#1{
  {\parindent \z@ \raggedright \normalfont
   \Huge\bfseries \thechapter. #1
   \par\nobreak
   \vskip 20\p@
}}
\def\@makeschapterhead#1{
  {\parindent \z@ \raggedright \normalfont
   \Huge\bfseries #1
   \par\nobreak
   \vskip 20\p@
}}
\makeatother

% Toto makro definuje kapitolu, která není očíslovaná, ale je uvedena v obsahu.
\def\chapwithtoc#1{
\chapter*{#1}
\addcontentsline{toc}{chapter}{#1}
}

\begin{document}

% Trochu volnější nastavení dělení slov, než je default.
\lefthyphenmin=2
\righthyphenmin=2

% Definice a věty
\newtheorem{theorem}{Theorem}[chapter]
\newtheorem{define}[theorem]{Definition}	% Definice číslujeme
%\newtheorem*{define}{Definition} 	% Definice nečíslujeme
\newtheorem{example}[theorem]{Example}	

\newcommand{\tbd} {\emph{\color{red} TBD}}


%%% Titulní strana práce

\pagestyle{empty}
\begin{center}

\large

Charles University in Prague

\medskip

Faculty of Mathematics and Physics

\vfill

{\bf\Large BACHELOR THESIS}

\vfill

\centerline{\mbox{\includegraphics[width=60mm]{logo.eps}}}

\vfill
\vspace{5mm}

{\LARGE Robin Pokorný}

\vspace{15mm}

% Název práce přesně podle zadání
{\LARGE\bfseries Properties of Metric Spaces by Means of Convergence}

\vfill

% Název katedry nebo ústavu, kde byla práce oficiálně zadána
% (dle Organizační struktury MFF UK)
Department of Mathematical Analysis

\vfill

\begin{tabular}{rl}

Supervisor of the bachelor thesis: & prof. RNDr. Miroslav Hušek, DrSc. \\
\noalign{\vspace{2mm}}
Study programme: & Mathematics \\
\noalign{\vspace{2mm}}
Specialization: & General Mathematics \\
\end{tabular}

\vfill

% Zde doplňte rok
Prague 2012

\end{center}

\newpage

%%% Následuje vevázaný list -- kopie podepsaného "Zadání bakalářské práce".
%%% Toto zadání NENÍ součástí elektronické verze práce, nescanovat.

%%% Na tomto místě mohou být napsána případná poděkování (vedoucímu práce,
%%% konzultantovi, tomu, kdo zapůjčil software, literaturu apod.)

\openright

\noindent
Acknowledgement (?).

\newpage

%%% Strana s čestným prohlášením k bakalářské práci

\vglue 0pt plus 1fill

\noindent
I declare that I carried out this bachelor thesis independently, and only
with the cited sources, literature and other professional sources.

\medskip\noindent
I understand that my work relates to the rights and obligations under
the Act No. 121/2000 Coll., the Copyright Act, as amended, in particular the fact
that the Charles University in Prague has the right to conclude a license agreement
on the use of this work as a school work pursuant to Section 60 paragraph 1 of the Copyright Act.

\vspace{10mm}

\hbox{\hbox to 0.5\hsize{%
In ........ date ............
\hss}\hbox to 0.5\hsize{%
Signature
\hss}}

\vspace{20mm}
\newpage

%%% Povinná informační strana bakalářské práce

\vbox to 0.5\vsize{
\setlength\parindent{0mm}
\setlength\parskip{5mm}

Název práce:
Vlastnosti metrických prostorů pomocí konvergence
% přesně dle zadání

Autor:
Robin Pokorný

Katedra:  % Případně Ústav:
Katedra matematické analýzy
% dle Organizační struktury MFF UK

Vedoucí bakalářské práce:
prof. RNDr. Miroslav Hušek, DrSc., Katedra matematické analýzy
% dle Organizační struktury MFF UK, případně plný název pracoviště mimo MFF UK

Abstrakt:
% abstrakt v rozsahu 80-200 slov; nejedná se však o opis zadání bakalářské práce

Klíčová slova:
% 3 až 5 klíčových slov

\vss}\nobreak\vbox to 0.49\vsize{
\setlength\parindent{0mm}
\setlength\parskip{5mm}

Title:
Properties of metric spaces by means of convergence

Author:
Robin Pokorný

Department:
Department of Mathematical Analysis
% dle Organizační struktury MFF UK v angličtině

Supervisor:
prof. RNDr. Miroslav Hušek, DrSc., Department of Mathematical Analysis
% dle Organizační struktury MFF UK, případně plný název pracoviště
% mimo MFF UK v angličtině

Abstract:
% abstrakt v rozsahu 80-200 slov v angličtině; nejedná se však o překlad
% zadání bakalářské práce

Keywords:
% 3 až 5 klíčových slov v angličtině

\vss}

\newpage

%%% Strana s automaticky generovaným obsahem bakalářské práce. U matematických
%%% prací je přípustné, aby seznam tabulek a zkratek, existují-li, byl umístěn
%%% na začátku práce, místo na jejím konci.

\openright
\pagestyle{plain}
\setcounter{page}{1}
\tableofcontents

%%% Jednotlivé kapitoly práce jsou pro přehlednost uloženy v samostatných souborech
\chapter*{Preface}
\addcontentsline{toc}{chapter}{Preface}

In the theory of metric spaces various properties are studied. These properties usually have numerous characterisations. We shall examine whether selected ones can be described using convergent sequences.

In the first chapter we remind the definition of a metric space and some of its interesting properties without using sequences. Stated theorems show equivalent conditions or basic behaviour. 

The second chapter introduces two abstract spaces: sequential and uniformly sequential spaces. Both of them provide a generalization of certain properties of sequences we examined in the first chapter.  We show their relations to metric spaces and to each other.

In the third chapter properties are divided into three parts. Countinuous mappings, topology, compactness, connectedness and separability are those which can be characterised using sequential spaces only. We also examine separability behaves slightly differently althrought equivalently in metric spaces. Total boundedness and completeness need more than this, we show that can be characterised using uniformly sequential spaces. The last part concerns properties which cannot be characterised using either of these structures, we show boundedness is one of them.
\chapter{Role of convergence in the theory of metric spaces}

\section{Sequences in metric spaces and their convergence}

We will briefly remind selected definitions and theorems, for proofs see \cite{cech66} or \cite{copson88}.
\begin{define}\label{def:mp}
Let $X$ be a non-empty set. Let $\varrho: X\times X \to \mathbb{R};$ be a mapping satisfying the following conditions:
\begin{enumerate}
	\item[(M1)] For $x,y\in X: \varrho(x,y)=0 \Leftrightarrow x = y$.
	\item[(M2)] For $x,y,z\in X: \varrho(x,z) \le \varrho(x,y) + \varrho(z,y)$.
\end{enumerate}
We say that $\varrho$ is a \emph{metric on $X$} and $(X,\varrho)$ is a \emph{metric space}.
\end{define}
The number $\varrho(x,y)$ is called the \emph{distance between x and y}.Condition \emph{(M1)} is called \emph{Identity of indiscernibles} and \emph{(M2)} is called \emph{Triangle inequality}.  It is an easy exercise to show that following holds:
\begin{enumerate}
	\item[(M3)] For $x,y\in X: \varrho(x,y)=\varrho(y,x)$.
\end{enumerate}
Sometimes \emph{(M2)} is writen with $\varrho(y,z)$ as the last item instead of $\varrho(z,y)$ but then \emph{(M3)} must be added to the conditions.

When we set $x=z$ in \emph{(M2)} it follows that $\varrho(x,y)\ge0$ for arbitrary points $x,y\in X$.

\begin{define}\label{def:allprp}
Let $(X,\varrho)$ be a metric space.

\begin{enumerate}[(i)]
	\item For $x_0\in X$ and $\varepsilon > 0$ the set $B(x_0,\varepsilon):=\{x\in X : \varrho(x_0,x)<\varepsilon\}$ is called the \emph{open ball with centre $x_0$ and radius $\varepsilon$}.
	\item Set $A\subset X$ is called \emph{open in $(X,\varrho)$} if $\forall x\in A\ \exists \varepsilon>0: B(x,\varepsilon)\subset A$. Set $B\subset X$ is called \emph{closed in $(X,\varrho)$} if $X\setminus B$ is open.
	\item Let $\sigma$ be a metric on $X$. We say $\varrho$ and $\sigma$ are \emph{equivalent} if the family of all sets open in $(X,\varrho)$ is the same as the family of all sets open in $(X,\sigma)$.
	\item For a non-empty $A\subset X$ the number $\mathrm{diam}(A):=\sup_{x,y\in A}\varrho(x,y)$ is called the diameter. We define $\mathrm{diam}(\emptyset):=0$. The space $(X,\varrho)$ is called \emph{bounded} if $\mathrm{diam}(X)<\infty$, otherwise it is called \emph{unbounded}.
	\item
\end{enumerate}
\end{define}


\begin{example}
	\tbd\ Discrete space
\end{example}

\section{Some known properties of sequences in metric spaces}

\begin{define}\label{def:consq}
Let $(X,\varrho)$ be a metric space, $\{x_n\}_{n=1}^\infty$ be a sequence in $X$. We say that \emph{$\{x_n\}_{n=1}^\infty$ converges to $x\in X$} (or $x$ is a limit of $\{x_n\}_{n=1}^\infty$) if $\lim_{n\to\infty}\varrho(x_n,x)=0$ and we will denote it as $\lim_{n\to\infty}x_n=x$ or $x_n\to x$.
\end{define}

We will sometimes leave the indexes of a sequence and thus write only $\{x_n\}$.

\begin{define}\label{def:causq}
Let $(X,\varrho)$ be a metric space, $\{x_n\}_{n=1}^\infty$ be a sequence in $X$. We say that \emph{$\{x_n\}_{n=1}^\infty$ is Cauchy} if
\[
	\forall \varepsilon\ \exists n_0\in \mathbb{N} \ \forall n,m \in \mathbb{N}; n,m\ge n_0:\ \varrho(x_n,x_m)<\varepsilon.
\]
\end{define}

\begin{theorem}[Properties of convergent sequencies] \label{th:mpseq}
Let $(X,\varrho)$ be a metric space, $\{x_n\}$ be a sequence in $X$. Then:
\begin{enumerate}[(i)]
	\item If $\{x_n\}\to x\in X$ and $\{x_n\}\to y\in X$ then $x=y$.
	\item If $\{x_n\}$ is convergent then it is bounded \tbd.
	\item Sequence $\{x_n\}_{n=1}^\infty$ converges to $x\in X$ if and only if any of its subsequences $\{x_{n_k}\}_{k=1}^\infty$ converges to $x$.
	\item Let $\sigma$ be a metric on $X$ equivalent with $\varrho$ and $x\in X$ then $\{x_n\}\to x$ in $(X,\varrho)$ if and only if $\{x_n\}\to x$ in$(X,\sigma)$.
\end{enumerate}
\end{theorem} 


\section{Historical notes}


\chapter{Sequential and uniformly sequential spaces}

\section{Family of convergent sequences on a set}

To study the role of convergence in metric spaces we take some properties of sequences which hold in metric spaces and create new space using these properties as conditions. It turns out that the following conditions \emph{(C1)} - \emph{(C4)} introduce a rich class of spaces yet relatively simple to explore. 

\begin{define}\label{def:fcs}
Let $X$ be a non-empty set. A \emph{\fcs\ on $X$} (denoted by $\Xi_X$) is a set of pairs $(\{x_n\},x)$ consisting of a sequence $\{x_n\}\subseteq X$ and a point $x\in X$ satisfying the following conditions:
\begin{enumerate}
	\item[(C1)] If $(\{x_n\},x)\in\Xi_X$ and $\{y_n\}$ is a subsequence of $\{x_n\}$ then $(\{y_n\},x)\in\Xi_X$.
	\item[(C2)] If $(\{x_n\},x)\in\Xi_X$, $(\{x_n\},y)\in\Xi_X$ then $x=y$.
	\item[(C3)] If  $(\{x_n\},x)\notin\Xi_X$ then there exists a subsequence $\{x_{n_k}\}^\infty_{k=1}$ such that for none of its subsequence $\{x_{n_{k_i}}\}$ is $(\{x_{n_{k_i}}\},x)\in\Xi_X$.
	\item[(C4)] If $x\in X$ then $(\{x\},x)\in\Xi_X$.
\end{enumerate}
We say $(X,\Xi_X)$ is a \emph{sequential space}.
\end{define}

In the next we will write $\Xi$ instead of $\Xi_X$ wherever the set $X$ is clear from the context. Unless stated otherwise $X$ will denote a non-empty set. When $(\{x_n\},x)\in\Xi$ we say that $\{x_n\}$ is  a \emph{convergent sequence}, $\{x_n\}$ \emph{converges} to $x$ or that $x$ is a limit of  $\{x_n\}$.

The properties in Definition \ref{def:fcs} can be restated as follows:
\begin{enumerate}
	\item[(C1')] A subsequence of a convergent sequence is convergent and converges to the same point.
	\item[(C2')] Every sequence has at most one limit.
	\item[(C3')] A sequence that is not converging to a point $x$ contains a subsequence such that any of its subsequences either converges to a different point (not to $x$) or does not converge at all.
	\item[(C4')] A constant sequence converges to the common point.
\end{enumerate}
The first two and the fourth properties are natural and follow from basic knowledge of convergence in metric spaces. To understand the condition \emph{(C3)} we need to remind that a sequence $\{x_n\}$ which does not converge to $x$ might include a subsequence $\{x_{n_k}\}$ which converges to $x$.

Our motivation to define \fcss\ was to generalise convergence in metric spaces. The following shows the relation between them.

\begin{define}\label{def:gen}
Let $(X,\varrho)$ be a metric space. We say $\Xi$ is \emph{generated by} $X$ if 
\[
	\Xi=\{(\{x_n\},x): x\in X, \forall n\in \mathbb{N}, \ x_n\in X, \lim_{n \to \infty} \varrho(x_n,x)=0\}.
\]
\end{define}

A \fcs\ generated by a metric space is well defined. We know that $\lim_{n\to\infty}\varrho(x_n,x)=0$ implies $\lim_{n\to\infty}\varrho(x_{n_k},x)=0$ for every subsequence $\{x_{n_k}\}$ of $\{x_n\}$. If $x_n\to x$ and $x_n\to y$ for some $x,y\in X$ we have
\[
	0\le\varrho(x,y)\le\varrho(x_n,x)+\varrho(x_n,y)\to 0
\]
as $n\to\infty$, and so $\varrho(x,y)=0$ which by \emph{(M1)} means $x=y$. Property \emph{(C3)} follows from Theorem \ref{th:mpseq}. And last, obviously $\{x\}\to x$. 

In metric spaces adding a finite number of members to a sequence as well as removing finite number of members from it does not affect the convergence of the sequence. The same holds for \fcss.

\begin{theorem} \label{th:eqseq}
Let $\Xi$ be a \fcs, $(\{x_n\},x)\in\Xi$ and $a\in X$. Let $\{y_n\}$ and $\{z_n\}$ be sequences defined as follows:
\begin{enumerate}[(i)]
	\item $y_1=a$, $y_{n+1}=x_n\quad \forall n\in \mathbb{N}$,
	\item $z_n=x_{n+1}\quad \forall n\in \mathbb{N}$.
\end{enumerate}
Then $(\{y_n\},x), (\{z_n\},x)\in\Xi$.
\end{theorem} 
\begin{proof}
Let for contradiction suppose $(\{y_n\},x)\notin\Xi$. Then using \emph{(C3)} we can find a subsequence $\{y_{n_k}\}$ such that for every its subsequence $(\{y_{n_{k_j}}\},x)\notin\Xi$. We can assume $\{y_{n_{k_j}}\}$ does not include the first member of $\{y_n\}$. Then $\{y_{n_{k_j}}\}$ is a subsequence of $\{x_n\}$ and by \emph{(C1)} we have $(\{y_{n_{k_j}}\},x)\in\Xi$ which is a contradiction.

Since $\{z_n\}$ is a subsequence of $\{x_n\}$ the conclusion follows from \emph{(C1)}.
\end{proof}

\begin{theorem} \label{th:intxi}
An intersection of \fcss\ on a set $X$ is a \fcs\ on $X$.
\end{theorem} 
\begin{proof}
Let $\Xi_\alpha$ be a \fcs\ on $X$ for every $\alpha\in A$, where $A$ is an arbitrary non-empty set. Let $(\{x_n\},x)\in \bigcap_{\alpha\in A}\Xi_\alpha$ and $\{y_n\}$ be a subsequence of $\{x_n\}$. Then for every $\alpha\in A$ we have $(\{x_n\},x)\in \Xi_\alpha$ so by \emph{(C1)} we have $(\{y_n\},x)\in \Xi_\alpha$. That is, $(\{y_n\},x)\in \bigcap_{\alpha\in A}\Xi_\alpha$.

Also for every $\alpha\in A$ and every $x\in X$ we have $(\{x\},x)\in \Xi_\alpha$ so $(\{x\},x)\in \bigcap_{\alpha\in A}\Xi_\alpha$.

Let $(\{x_n\},x)\not\in \bigcap_{\alpha\in A}\Xi_\alpha$ then there exists $\alpha_0\in A$ for which $(\{x_n\},x)\not\in \Xi_{\alpha_0}$. We find the subsequence $\{x_{n_k}\}$ from \emph{(C3)} for $\Xi_{\alpha_0}$ and let $\{x_{n_{k_i}}\}$ be its subsequence. Then $(\{x_{n_{k_i}}\},x)\not\in\Xi_{\alpha_0}$ therefore $(\{x_{n_{k_i}}\},x)\not\in \bigcap_{\alpha\in A}\Xi_\alpha$.

Let $(\{x_n\},x)\in \bigcap_{\alpha\in A}\Xi_\alpha$ and $y\in X,\ x\ne y$. Then for some $\alpha\in A$ we have $(\{x_n\},x)\in \Xi_\alpha$ and from \emph{(C2)} for $\Xi_\alpha$ we have $(\{x_n\},y)\not\in \Xi_\alpha$. Therefore $(\{x_n\},y)\not\in \bigcap_{\alpha\in A}\Xi_\alpha$.
\end{proof}

\begin{define}\label{def:xi0}
Let $X$ be a non-empty set. Symbol $\Xi_0$ will denote the smallest (in the meaning of inclusion) \fcs\ on X.
%for which the following holds: \[\Xi_0\supset\bigg\{(\{x\},x) : x\in X\bigg\}.\]
\end{define}

%The existence of such \fcs\ follows from the Zorn's lemma and Theorem \ref{th:intxi}.
The definition says that $\Xi_0$ consists of convergent sequences which are constant up to a finite number of members.

Since $\Xi_0$ is the smallest \fcs\ on the set it is wherefore interesting to see for which metric spaces $\Xi=\Xi_0$ holds. It is easy to observe that $\Xi$ generated by a metrically discrete space (in Example \ref{ex:msp}) has his property.


At first sight one might think that  $\Xi=\Xi_0$ implies the space $(X,\varrho)$ to be metrically discrete. The following shows a counterexample constructing a space that is not metrically discrete.
\begin{example}
	Suppose $X=\{\frac{1}{n}, n\in \mathbb{N}\}$, then $X\subset \mathbb{R}$ and with induced metric it forms a subspace of $\mathbb{R}$. As the sequence $\{\frac{1}{n}\}$ has a limit in $\mathbb{R}$ which is not included in $X$, every convergent sequence in $X$ has to be constant from some index. Therefore $\Xi=\Xi_0$.
\end{example}

Another idea we had was: If $\Xi=\Xi_0$ then either $X$ is metrically discrete or $X$ is not complete. The following example prooves this statement to be false.

\begin{example}
	Suppose $X=\{(n,0), n\in \mathbb{N}\}\cup\{(n,\frac{1}{n}), n\in \mathbb{N}\}$, then $X\subset \mathbb{R}^2$. Intuitive image is the set of natural numbers as a subset of $x$ axis united with the graph of $\{\frac{1}{n}\}$ in the plane. This space (with induced metric) is complete and every convergent sequence is constant from some index.
\end{example}


\begin{theorem} \label{th:genxi0}
Let $\Xi$ be generated by a metric space $(X,\varrho)$ then the following are equivalent:
\begin{enumerate}[(i)]
	\item $\Xi=\Xi_0$;
	\item $\forall x\in X,\ \exists \varepsilon_x>0,\ \forall y\in X, x\neq y: \varrho(x,y)>\varepsilon_x$. (We say $(X,\varrho)$ is \emph{topologically discrete}.)
\end{enumerate}
\end{theorem} 

\begin{proof}
Let $X$ be a topologically discrete metric space, $\Xi$ generated by $X$ and for contradiction suppose there exists a sequence $\{x_n\}$ and $x\in X$ such that $(\{x_n\},x)\in\Xi\setminus\Xi_0$. Then from (ii) we find $\varepsilon_x$. Because $\lim_{n\to\infty}\varrho(x_n,x)=0$ there is $n_0\in\mathbb{N}$ for which $\varrho(x_{n_0},x)\le\frac{1}{2}\varepsilon_x$ and $x_{n_0}\ne x$ (since $(\{x_n\},x)\notin\Xi_0$). This is a contradiction with $X$ being topologically discrete.

Let $\Xi=\Xi_0$ and for contradiction suppose that for some $x\in X$ we have
\[
	\forall \varepsilon>0,\ \exists y\in X, x\ne y: \varrho(x,y)<\varepsilon.
\]
For $n\in \mathbb{N}$ we set $\varepsilon_n:=\frac{1}{n}$ and find $y_n\ne x$ such that $\varrho(x,y_n)<\varepsilon_n$. Then $\{y_n\}\to x$ which is a contradiction since $\{y_n\}$ is not constant.
\end{proof}

\section{Uniformly sequential structure on a set}

%\begin{define}{Properties needed later \tbd} \label{def:cofin}
%\begin{enumerate}[(i)]
%	\item $A'$ is cofinal in $A$.
%\end{enumerate}
%\end{define}

% \{x_n\} 

\begin{define} \label{def:unistr}
Let $X$ be a non-empty set.  A \emph{\uss\ on $X$} is an equivalence $\sim$ on sequences in $X$ satisfying the following conditions:
\begin{enumerate}
	\item[(U1)] If $\{x_n\} \sim \{y_n\}$ and $\{x_{n_k}\}, \{y_{n_k}\}$ are subsequences then $\{x_{n_k}\} \sim \{y_{n_k}\}$.
	\item[(U2)] If $x_n=x, y_n=y\ \forall n\in \mathbb{N}$ for some $x,y\in X$ and $\{x_n\} \sim \{y_n\}$ then $x=y$.
	\item[(U3)] If $\{x_n\} \not\sim \{y_n\}$ then there exist subsequences $\{x_{n_k}\}, \{y_{n_k}\}$ such that for none of their subsequences $\{x_{n_{k_i}}\}, \{y_{n_{k_i}}\}$ is $\{x_{n_{k_i}}\}\sim \{y_{n_{k_i}}\}$.
	\item[(U4)] If $x\in X$ then $\{x\} \sim \{y\}$. (Reflexivity) 
\end{enumerate}
We say $(X,\sim)$ is a \emph{uniformly sequential space}. If $\{x_n\} \sim\{y_n\}$ we say  \emph{$\{x_n\} $ is adjacent to $\{y_n\}$}.
\end{define}

\begin{theorem} \label{th:unitoseq}
	Let $X$ have a \uss. Then
\[
	\Xi_\sim:=\bigg\{(\{x_n\},x): \{x_n\}\sim \{x\}\bigg\}
\]
is a \fcs\ on $X$.
\end{theorem}
\begin{proof}
Let $\{x_n\}\sim \{x\}$ and $\{x_{n_k}\}$ be subsequence of $\{x_n\}$. We know every subsquence of a constant sequence is again a constant sequence. So from \emph{(U1)} we have $\{x_{n_k}\}\sim \{x\}$.

Let $x,y\in X$. Let $\{x_n\} \sim \{x\}$ and $\{x_n\} \sim \{y\}$. Then from transitivity of $\sim$ we obtain $\{x\} \sim \{y\}$ and \emph{(U2)} gives us $x=y$.

Let $\{x_n\} \not\sim \{x\}$. Then \emph{(C3)} follows directly from \emph{(U3)}.

Because of \emph{(U4)} we have $(\{x\},x)\in\Xi_\sim$.
\end{proof}

\begin{theorem}\label{th:adjprop}
Adjancent sequences in a metric space $(X,\varrho)$ satisfy conditions \emph{(U1)-(U4)}.
\end{theorem}
\begin{proof}
From Theorem \ref{th:adjquiv} we know  $\sim_\varrho$ is an equivalence.

Let $x_n=x, y_n=y\ \forall n\in \mathbb{N}$ for some $x,y\in X$ and $\{x_n\} \sim_\varrho \{y_n\}$. If $x\ne y$ then from \emph{(M1)} we have $\varrho(x,y)>0$ so $x=y$.

Let $\{x_n\} \sim_\varrho \{y_n\}$ and $\{x_{n_k}\}, \{y_{n_k}\}$ be subsequences. Then clearly $\{x_{n_k}\}\sim_\varrho \{y_{n_k}\}$ .

Let $\{x_n\} \not\sim_\varrho \{y_n\}$ then there exists $\varepsilon>0$ such that for infinitely many $n \in \mathbb{N}$ we have $\varrho(x_n,y_n)\ge\varepsilon$, these introduce a subsequence. It is easy to observe it has desired properties.
\end{proof}

We are not able to characterise a Cauchy sequence in terms of simple convergence. It turns out that this problem is the main motivation to develop uniformly sequential spaces in which we can define this property.

\begin{define}\label{def:ucauch}
Let $X$ have a \uss. We say a sequence $\{x_n\}$ in $X$ is \emph{Cauchy} if it is adjacent to all its subsequences.
\end{define}

As we would expect every convergent sequence is Cauchy.

\begin{theorem}\label{th:cauchconvg}
Let $X$ have a \uss. If $(\{x_n\}, x)\in\Xi_\sim$ then $\{x_n\}$ is Cauchy.
\end{theorem}
\begin{proof}
Let $\{x_{n_k}\}$ be a subsequence of $\{x_n\}$. Then from \emph{(C1)} we have $(\{x_{n_k}\}, x)\in\Xi_\sim$. That is $\{x_n\} \sim \{x\}$ and $\{x_{n_k}\} \sim \{x\}$ and the transitivity of $\sim$ gives us $\{x_{n_k}\} \sim \{x_n\}$.
\end{proof}

The relation of a Cauchy sequence in a uniformly sequential space and a Cauchy sequence in a metric space is shown later (Theorem \ref{th:scomplcompl}).

%Poznámka: (Více na Császar: GT, 1978 - str. 153 for totally bounded uniform spaces, str. 139 for uniform spaces)





















\chapter{Properties defined by convergence} 

In this chapter we will use the following convention: when we say a uniformly sequential space $(X,\sim)$ has a property defined for a sequential space we mean that $\Xi_\sim$ has this property.

\section{Continuous and uniformly continuous mappings}

\begin{define}\label{def:cont}
Let $\Xi_X$ and $\Xi_Y$ be two \fcss\ on sets $X$ and $Y$. We say $f:X\to Y$ is a \emph{continuous mapping $\Xi_X$ to $\Xi_Y$} if $(\{x_n\},x)\in\Xi_X$ implies that $\left(\{f(x_n)\},f(x)\right)\in\Xi_Y$.
\end{define}

%To distinguish among Definitions \ref{def:contf} and \ref{def:cont} of a continuous mapping we will sometimes write $f:(X,\varrho)\to (Y, \sigma)$, $f:\Xi_X\to \Xi_Y$. The following theorem states that, in fact, both definitions are the same.
The following theorem is often covered in basic course of mathematical analysis. We include it for completeness.

\begin{theorem} \label{th:contpr}
Let $\Xi_X$ and $\Xi_Y$ be two \fcss\ generated by metric spaces $(X,\varrho)$ and $(Y,\sigma)$. Let $f:X\to Y$ be a mapping. Then the following are equivalent:
\begin{enumerate}[(i)]
	\item $f:(X,\varrho)\to (Y, \sigma)$ is continuous;
	\item $f:\Xi_X\to \Xi_Y$ is continuous.
\end{enumerate}
\end{theorem} 
\begin{proof}
Let $x\in X$ be a fixed point.

Let condition \emph{(i)} hold and $x_n\to x$ (that is, $(\{x_n\},x)\in\Xi_X$). For $\varepsilon>0$ we find $\delta>0$ from Definition  \ref{def:contf}. Then there exists $n_\varepsilon$ such that for $\forall n\ge n_\varepsilon$ we have $\varrho(x_n, x)<\delta$ and then $\sigma(f(x_n), f(x))<\varepsilon$. Hence $f(x_n)\to f(x)$ in $Y$, that is, $(\{f(x_n)\},f(x))\in\Xi_Y$.

Let now condition \emph{(ii)} hold and for contradiction suppose that \emph{(i)} does not. Then
\[
	\exists \varepsilon>0\ \forall\delta>0\ \forall y\in X: \varrho(x,y)<\delta \Rightarrow \sigma(f(x), f(y))\ge\varepsilon.
\]
We fix this $\varepsilon$ and find a sequence $\{x_n\}$ such that $\varrho(x_1,x)<1$ and $\varrho(x_{n+1},x)<min(\frac{1}{n},\varrho(x_n,x))$ for $n\in \mathbb{N}$. Then $x_n\to x$ hence from \emph{(ii)} $f(x_n)\to f(x)$. This is not possible since 
$\sigma(f(x_n), f(x))\ge\varepsilon$ for $n\in \mathbb{N}$.
\end{proof}

\begin{define}\label{def:uncont}
Let $(X,\sim_X)$ and $(Y,\sim_Y)$ be two uniformly sequential spaces. We say $f:X\to Y$ is a \emph{uniformly continuous mapping $\sim_X$ to $\sim_Y$} if $\{x_n\}\sim_X\{y_n\}$ implies that $\{f(x_n)\}\sim_Y\{f(y_n)\}$.
\end{define}

\begin{theorem} \label{th:contpr}
Let $\sim_X$ and $\sim_Y$ be two \uss{}s\ generated by metric spaces $(X,\varrho)$ and $(Y,\sigma)$. Let $f:X\to Y$ be a mapping. Then the following are equivalent:
\begin{enumerate}[(i)]
	\item $f:(X,\varrho)\to (Y, \sigma)$ is uniformly continuous;
	\item $f:(X,\sim_X)\to (Y,\sim_Y)$ is uniformly continuous.
\end{enumerate}
\end{theorem} 
\begin{proof}
Let \emph{(i)} hold and $\{x_n\}\sim_X\{y_n\}$. Since $\lim_{n\to\infty}\varrho(x_n,y_n)=0$ we have
\[
	\forall \delta>0\ \exists n_0\in\mathbb{N}\ \forall n\in\mathbb{N}, n\ge n_0: \varrho(x_n.y_n)<\delta.
\]
Let $\varepsilon>0$. We find $\delta$ from the definition of \emph{(i)}, then we find $n_0$ from the previous formula. We obtain
\[
	\forall n\in\mathbb{N}, n\ge n_0: \sigma(f(x_n),f(y_n))<\varepsilon,
\]
so $\lim_{n\to\infty}\sigma(f(x_n),f(y_n))=0$.

Let \emph{(i)} does not hold. Than there exists $\varepsilon>0$ such that for every $n\in\mathbb{N}$ we can find $x_n,y_n\in X$, $\varrho(x_n,y_n)<\frac{1}{n}$ and $\sigma(f(x_n),f(y_n))>\varepsilon$. That is, $\{x_n\}\sim_X\{y_n\}$ but $\{f(x_n)\}\not\sim_Y\{f(y_n)\}$.
\end{proof}

\section{Open and closed sets in sequential spaces and related topological properties}

If we were able to define a topology on $\Xi$ we would obtain many interesting properties. In this section we will try to achieve this.

With knowledge of the form of closed sets obtained from Theorem \ref{th:mpseq}\emph{(vi)} it seems natural to define a closure using similar approach and then use Theorem \ref{th:ico}.

\begin{define}\label{def:po}
Let $\Xi$ be a \fcs\ on $X$, $A\subset X$ and $x\in X$ then we say $x\in c(A)$ if $\exists\ (\{x_n\}, x)\in\Xi: \{x_n\}\subset A$. That is, a point lies in the closure of a set if there is a sequence in the set converging to the point.
\end{define}

However operator $c$ defined above is not a closure operator. It is easy to show that conditions \emph{(i)}-\emph{(iii)} in Theorem \ref{th:pco} hold, but generally \emph{(iv)} does not. In the following example we will show a space in which \emph{(iv)} does not hold. Please note that when $\Xi$ is generated by a metric space $c$ is a closure operator.

\begin{example}
	Let $X=\{f, f:\mathbb{R}\to \mathbb{R}\}$ be a space of all real functions on real numbers, $\{f_n\}\subset X$ a sequence of functions and $f\in X$ a function. We say $(\{f_n\}, f)\in\Xi$ if
\[
	\forall\ x\in \mathbb{R}: \lim_{n\to\infty}f_n(x)=f(x).
\]
Such convergence is usually called pointwise convergence.

Clearly $(\{f\},f)\in\Xi$. When $\{f_{n_k}\}\subset X$ is a subsequence of $\{f_n\}$ then for any $x\in \mathbb{R}$ $\{f_{n_k}(x)\}$ is a subsequence of $\{f_n(x)\}$. As these are sequences of real numbers we have $\lim_{n\to\infty}f_n(x)=\lim_{k\to\infty}f_{n_k}(x)$ and $(\{f_{n_k}\}, f)\in\Xi$. If $(\{f_n\}, f)\notin\Xi$ then there is $x_0\in \mathbb{R}$ for which $\lim_{n\to\infty}f_n(x_0)\ne f(x_0)$. And again since those are real sequences we easily obtain the remaining conditions and therefore $\Xi$ is a \fcs\ on $X$.

Now let $C$ be a set of all real continuous functions of real variable. Then $C\subset X$. Applying operator $c$ from Definition \ref{def:po} we have that $c(C)$ are continuous functions and functions of Baire class 1. Accordingly $c(c(C))$ are continuous functions, functions of Baire class 1 and functions of Baire class 2. It is known that Baire class 2 set in non-empty and therefore $c(C)\ne c(c(C))$.
\end{example}

Operator that satisfies all conditions but \emph{(iv)} is sometimes called a \emph{Čech closure operator}. 

\begin{define}\label{def:io}
Let $\Xi$ be a \fcs\ on $X$, $A\subset X$ and $x\in X$ then we say $x\in \mathrm{Int} A$ if $\forall\ (\{x_n\}, x)\in\Xi\ \exists \{x_{n_k}\}$ subsequence of $\{x_n\}: \{x_{n_k}\}\subset A$. Set $A$ is \emph{open in $(X,\Xi)$} if $A=\mathrm{Int} A$. That is, a set is open when all sequences converging to a point in the set eventually lie in the set.

A set $A\subset X$ is \emph{closed in $(X,\Xi)$} if $X\setminus A$ is open in $(X,\Xi)$.
\end{define}

To show that our first guess in Definition \ref{def:po} can lead to a topology owe introduce the next theorem, which we state without proof.

\begin{theorem} \label{th:topxi}
Operator $c_{\omega_1}$ defined by a transfinite induction
\[
	c_\alpha(A):=c\left(\bigcup_{\beta<\alpha} c_\beta(A)\right),
\]
where $\alpha$ is an ordinal number, $A\subset X$ and $c$ is operator from Definition \ref{def:po}, is a Kuratowski closure operator. That is, it meets the condition \emph{(iv)} in Theorem \ref{th:pco}:  $c(c_{\omega_1}(A))=c_{\omega_1}(A)$. Moreover $c_{\omega_1}$ generates the same closed sets (and therefore the same topology) as in Definition \ref{th:topxi}.
\end{theorem}

\section{Complete and totally bounded spaces}

\begin{define}\label{def:scompl}
	Let $X$ have a \uss. We say $X$ is \emph{complete} if for every Cauchy sequence $\{x_n\}$ in $X$ we have $(\{x_n\}, x)\in\Xi_\sim$ for some $x\in X$.
\end{define}

Completeness is very closely related to Cauchyness of sequences. That is why we need to define it using uniformly sequential spaces. The definition follows the definition of complete metric space. The next theorem therefore seems natural.

\begin{theorem} \label{th:scomplcompl}
	Let $(X,\sim_\varrho)$ be a uniformly sequential space generated by metric space $(X,\varrho)$. Then $(X,\varrho)$ is complete if and only if $(X,\sim_\varrho)$ is complete.
\end{theorem}
\begin{proof}
It is sufficient to show that a sequence $\{x_n\}$ is Cauchy in $(X,\varrho)$ if and only if it is Cauchy in $(X,\sim)$.

Let $\varepsilon>0$, let $\{x_n\}$ be a Cauchy sequence in $(X,\varrho)$ and $\{x_{n_k}\}$ be its subsequence. Because $\{x_n\}$ is Cauchy we find $m_0$ from Definition \ref{def:seqprp}. That is, for every $n,m\in \mathbb{N},\ n,m\ge m_0$ we have $\varrho(x_n,x_m)<\varepsilon$. Specially $\lim_{k\to\infty}\varrho(x_k, x_{n_k})=0$.

Let us suppose that there exists $\{x_{n_k}\}$ such that $\{x_n\}\not\sim_\varrho\{x_{n_k}\}$. That is,
\[
	\exists\varepsilon>0\ \forall k_0\in \mathbb{N}\ \exists k\in \mathbb{N}, k\ge k_0: \varrho(x_k, x_{n_k})\ge \varepsilon.
\]
Then if we choose $n=n_k, m=k$ in the definition of Cauchy sequence in metric space we see that $\{x_n\}$ is not Cauchy.
\end{proof}

\begin{define}\label{def:stb}
	Let $X$ have a \uss. We say $X$ is \emph{totally bounded} if for every sequence $\{x_n\}$ in $X$ there exists a subsequence $\{x_{n_k}\}$ which is Cauchy.
\end{define}

It is uncommon to characterise totally bounded metric spaces using this or similar condition. It is however equivalent as the next shows us.

\begin{theorem} \label{th:stbtb}
	Let $(X,\sim_\varrho)$ be a uniformly sequential space generated by a metric space $(X,\varrho)$. Then $(X,\varrho)$ is totally bounded if and only if $(X,\sim_\varrho)$ is totally bounded.
\end{theorem}
\begin{proof}
Let $(X,\varrho)$ be totally bounded and $\{x_{0,n}\}$ be a sequence in $X$. Then for every $m\in \mathbb{N}$ there exists a finite family of open balls $B_1, \dots, B_M$ with radius $\frac{1}{m}$ such that $X=\bigcup_{i=1}^MB_i$. Since a sequence is infinite there exists a ball which contains an infinite number of members of sequence $\{x_{m-1,n}\}$. We choose one such ball and denote $\{x_{m,n}\}$ a subsequence of $\{x_{m-1,n}\}$ which consists of points in the ball only. Then $\{x_{n,n}\}$ is Cauchy because for every $m\in \mathbb{N},\ n>m$ we have $\varrho(x_{n,n},x_{m,m})<\frac{2}{m}$.

To prove the sufficiency let us suppose that $(X,\varrho)$ is not totally bounded. That is, there exists $\varepsilon>0$ for which no finite family of open balls with diameter $\varepsilon$ covers whole $X$. We select an arbitrary point $x_1\in X$. Then for $n\in \mathbb{N}$ we find $x_{n+1}\in X\setminus\bigcup_{i=1}^n B(x_i, \varepsilon)\ne\emptyset$. Then any subsequence of $\{x_n\}$ is not Cauchy because for every $n, m\in \mathbb{N},\ n\ne m$ we have $\varrho(x_n,x_m)\ge\varepsilon$, so $(X,\sim)$ is not totally bounded.
\end{proof}

\section{Compact spaces}

Characterisation of compactness in metric space in term of convergence is widely known. We use this principle to define compact sequential space.

\begin{define}\label{def:cmp}
We say a sequential space \emph{$(X,\Xi)$ is compact} if for any arbitrary sequence $\{x_n\}\subset X$ there exists a subsequence $\{x_{n_k}\}$ and a point $x\in X$ such that $(\{x_{n_k}\}, x)\in\Xi$.
\end{define}

Direct proof of equivalence is covered in basic courses. We present a proof which follows from previously showed relation between total boundedness and completeness in metric and uniformly sequential spaces. The following theorem together with Theorems \ref{th:cmpcmpltb}, \ref{th:scomplcompl} and \ref{th:stbtb} give us the equivalence indirectly.

\begin{theorem} \label{th:comptbcompl}
A uniformly sequential space $(X,\sim)$ is compact if and only if it is totally bounded and complete.
\end{theorem}
\begin{proof}
Let $(X,\sim)$ be totally bounded and complete. Let $\{x_n\}$ be an arbitrary sequence in $X$. Because $(X,\sim)$ is totally bounded we can find a subsequence $\{x_{n_k}\}$ which is Cauchy. Because $(X,\sim)$ is complete we can find a point $x\in X$ such that $(\{x_{n_k}\}, x)\in\Xi_\sim$.

Let $(X,\sim)$ be compact. From Theorem \ref{th:cauchconvg} we have that is is  totally bounded. Let $\{x_n\}$ be a Cauchy sequence, we can find a subsequence $\{x_{n_k}\}$ and a point $x\in X$ such that $(\{x_{n_k}\}, x)\in\Xi_\sim$. That is, $\{x_{n_k}\}\sim\{x\}$ and since $\{x_n\}$ is Cauchy we have $\{x_n\}\sim\{x_{n_k}\}$, from transitivity of $\sim$ we obtain $\{x_n\}\sim\{x\}$. So $(\{x_n\}, x)\in\Xi_\sim$ and $(X,\sim)$ is complete.
\end{proof}

\begin{corollary} \label{cr:compseqsomp}
Let $(X,\varrho)$ be a metric space and $\Xi$ is generated by $(X,\varrho)$. Then $(X,\varrho)$ is compact if and only if $(X,\Xi)$ is compact.
\end{corollary}

\begin{theorem} \label{th:contucont}
Let $(X,\sim_X), (Y,\sim_Y)$ be two uniformly sequential spaces, $f:X\to Y$ be a continuous mapping and $(X,\sim_X)$ be compact. Then $f$ is uniformly continuous.
\end{theorem}
\begin{proof}
Let $\{x_n\} \sim_X\{y_n\}$ and let us for contradiction suppose that $\{f(x_n)\} \not\sim_Y\{f(y_n)\}$. Then from \emph{(U3)} we find subsequences $\{f(x_{n_k})\}$ and $\{f(y_{n_k})\}$ such that for none their subsequences we have $\{f(x_{n_{k_i}})\} \sim_Y\{f(y_{n_{k_i}})\}$. Because $X$ is compact we can find subsequences $\{x_{n_{k_i}}\}$ of $\{x_{n_k}\}$, $\{y_{n_{k_i}}\}$ of $\{y_{n_k}\}$ and points $x,y\in X$ such that $(\{x_{n_{k_i}}\},x)\in\Xi_{\sim_X}$ and $(\{y_{n_{k_i}}\},y)\in\Xi_{\sim_X}$. Since $\{x_n\} \sim_X\{y_n\}$ we have $\{x_{n_k}\} \sim_X\{y_{n_k}\}$ and therefore $x=y$. Mapping $f$ is continuous so $(\{f(x_{n_{k_i}})\},f(x))\in\Xi_{\sim_Y}$ and $(\{f(y_{n_{k_i}})\},f(x))\in\Xi_{\sim_Y}$. That is, $\{f(x_{n_{k_i}})\}\sim_Y\{f(x)\}$ and $\{f(y_{n_{k_i}})\}\sim_Y\{f(x)\}$ which gives us $\{f(x_{n_{k_i}})\}\sim_Y\{f(y_{n_{k_i}})\}$ which is the contradiction.
\end{proof}

The following example proves that the implication in the previous theorem can not be reversed.

\begin{example}\label{ex:madn}
We say two infinite sets are \emph{almost disjoint} if their intersection is finite. Let us assume an infinite maximal family of almost disjoint subsets of $\mathbb{N}$ and denote it as $\mathrm{MAD}(\mathbb{N})$.
	
For a reader's image of some infinite family of almost disjoint subsets of $\mathbb{N}$ we will show a construction of one. Let $f:\mathbb{N}\to\mathbb{Q}$ be a bijection of natural numbers onto rational numbers. For every irrational $x\in\mathbb{R}\setminus\mathbb{Q}$ we will choose one rational sequence $\{r_n\}$ such that $r_n\to x$. Then $\mathcal{S}_x:=f^{-1}[\{r_n\}]\subset \mathbb{N}$ is a preimage of that sequence. Then  $\{\mathcal{S}_x: x\in\mathbb{R}\setminus\mathbb{Q}\}$ is an infinite family of almost disjoint subsets of $\mathbb{N}$ because if two sets had an infinite intersection the corresponding rational sequences would have an infinite intersection too and since they are both convergent it follows they would converge to the same point.

Please note that using this construction we do not obtain a $\mathrm{MAD}(\mathbb{N})$ as for example the preimage of the set of all even numbers is almost disjoint with all constructed sets.

We denote $\mathrm{MAD}(\mathbb{N})=\{\mathcal{S}_\alpha: \alpha\in A\}$ for some set $A$. Let $X:=\mathbb{N}\cup\mathrm{MAD}(\mathbb{N})$ with the smallest sequential structure such that
\[
	\mathcal{S}_\alpha\to\{\mathcal{S}_\alpha\},\quad \{\mathcal{S}_\alpha\}\in \mathrm{MAD}(\mathbb{N}).
\]
This condition can be rephrased: members of $\mathcal{S}_\alpha$ as points in $\mathbb{N}$ converge to $\mathcal{S}_\alpha$ as point in $\mathrm{MAD}(\mathbb{N})$.
Then every continuous mapping to $\mathbb{R}$ is bounded but the space is evidently not compact.

To prove this let $f:X\to\mathbb{R}$ be a continuous unbounded mapping. That is, there exists a sequence $\{x_n\}$ for which $f(x_n)\to\infty$. Now we examine two cases.

First, infinitely many members of $\{x_n\}$ lie in $\mathbb{N}$, so we can find a subsequence $\{x_{n_k}\}\subset\mathbb{N}$. Then there exists $\mathcal{S}_\alpha\in \mathrm{MAD}(\mathbb{N})$ such that $\{x_{n_k}\}\to\mathcal{S}_\alpha$. Then we have $\{f(x_{n_k})\}\to f(\mathcal{S}_\alpha)$ and $\{f(x_{n_k})\}\to \infty$ which is a contradiction.

Second, only finite number of members of $\{x_n\}$ lie in $\mathbb{N}$. Without loss of generality we suppose that $\{x_n\}=\{\mathcal{S}_{\alpha_n}\}\subset\mathrm{MAD}(\mathbb{N})$. For every $n\in\mathbb{N}$ we can find $y_n\in\mathbb{N}$ such that $|f(y_n)-f(\mathcal{S}_{\alpha_n})|\le1$. The first case yields $\{f(y_n)\}$ is bounded and so $\{f(\mathcal{S}_{\alpha_n})\}$ is bounded.
\end{example}


\section{Bounded spaces}

Diameter of a set extremely depends on used metric. At first, it might seem natural that when a set has a finite (infinite) diameter it will have it finite (infinite) in every equivalent metric. The following theorem uses quite known definition of equivalent metrics to prove otherwise.

\begin{theorem} \label{th:unbdtobd}
When $(X,\varrho)$ is an unbounded metric space then there exists a metric $\sigma$ on $X$ that is equivalent with $\varrho$ and $(X,\sigma)$ is bounded.
\end{theorem}
\begin{proof}
For $x,y\in X$ we define
\[
	\sigma_1(x,y):=\frac{\varrho(x,y)}{1+\varrho(x,y)},\qquad\sigma_2(x,y):=\min\{\varrho(x,y), 1\}.
\]
See \cite[p.~22]{copson88} for $\sigma_1$ and \cite[p.~250]{engelking89} for $\sigma_2$ for proofs that they satisfy the conditions for a metric. Then evidently both $\sigma_1, \sigma_2$ are equivalent with $\varrho$ and are bounded.
\end{proof}

The previous theorem not only shows a diameter of a space in an equivalent metric can be finite, it also shows it can be any positive real number $a>0$ we want. We just take $\sigma(x,y):=a\cdot\sigma_i(x,y)$ where $\sigma_i$ is from Theorem \ref{th:unbdtobd} and $i=1$ or $i=2$.

This result is enough to answer the problem. We also show the not-so-known theorem that under some quite unrestrictive conditions a space with finite diameter in some metric can have the infinite diameter in an other equivalent metric. The proof is constructive, that is, we give exact form of such metric.

\begin{theorem} \label{th:bdtounbd}
When $(X,\varrho)$ is a non-compact bounded metric space then there exists a metric $\sigma$ on $X$ that is equivalent with $\varrho$ and $(X,\sigma)$ is unbounded.
\end{theorem}
\begin{proof}
Had we constructed a function $f: X\to \mathbb{R}$ continuous and unbouded we would define
\[
	\sigma(x,y):=\varrho(x,y)+|f(x)-f(y)|.
\]

Evidently $\sigma(x,x)=0$. We have $\sigma(x,z) \le \sigma(x,y)+\sigma(z,y)$ because when rewitten it is
\[
	\varrho(x,z)+|f(x)-f(z)| \le \varrho(x,y)+\varrho(z,y)+|f(x)-f(y)|+|f(z)-f(y)|.\nonumber
\]
This holds because $|f(y)-f(z)|=|f(z)-f(y)|$, the triangle inequality holds for $\varrho$ and
\[
	|f(x)-f(z)|=|f(x)-f(y)+f(y)-f(z)|\le |f(x)-f(y)|+|f(y)-f(z)|.
\]

Let $\{x_n\}\subset X$ and $x_n\to x\in X$ in metric $\varrho$. Since $f$ is continuous we know
\[
	\lim_{n\to\infty}|f(x_n)-f(x)|=0
\]
and therefore $x_n\to x$ also in metric $\sigma$.

Let now $\{y_n\}\subset X$ and $y_n\to y\in X$ in metric $\sigma$. Then from
\[
	0\le\lim_{n\to\infty} \varrho(y_n,y)\le\lim_{n\to\infty} \left(\varrho(y_n,y) + |f(y_n)-f(y)|\right)=\lim_{n\to\infty} \sigma(y_n,y)=0
\]
follows  $y_n\to y$ also in metric $\varrho$. That is, $\sigma$ and $\varrho$ are equivalent and $(X,\sigma)$ is unbounded.

When a metric space is non-compact it means there exists a sequence for which no subsequence converges, let $\{x_n\}$ be such sequence. Then $\{x_n\}$ is closed in $X$. Without loss of generality we suppose that $x_n\ne x_m$ if $n\ne m$. 

Then for every $n\in \mathbb{N}$ there exists $a_n>0$ such that $\forall m\in\mathbb{N}, n\ne m: \varrho(x_n,x_m)\ge 3a_n$. 
We define a function $f: X\to \mathbb{R}$:
\[
	f(x)=\left\{ \begin{array}{ll}n\cdot\varrho(x, X\setminus B(x_n, a_n)), & \mathrm{if}\ x\in B(x_n, a_n)\ \mathrm{for\ some}\ n\in\mathbb{N}; \\ 0, & \mathrm{othervise}. \end{array} \right.
\]
We can easily see that $f$ is continuous and unbounded.
\end{proof}

The previous theorems show that we (generally) can not distinguish between $\Xi$ generated by a bounded and an unbounded metric on the same set. Because of this we do not define any kind of boundedness of sequential spaces.

As a side product we obtain another interesting characterisation of compactness in metric spaces.

\begin{corollary} \label{cr:bdcomp}
A metric space $(X,\varrho)$ is bounded in every equivalent metric if and only if it is compact.
\end{corollary}

\section{Connected spaces}

Since we introduced topology on sequential spaces we could delegate the problem of characterisation of connectedness to it. It is however useful to define connected spaces with continuous mapping.

In the following we take $D=\{0,1\}$ with $(\Xi_D)_0$, that is, a two-point space with discrete (in meaning of Theorem \ref{th:genxi0}) sequential structure.

\begin{define}\label{def:conn}
A sequential space $(X,\Xi)$ is \emph{connected} if every continuous mapping $f:X\to D$ is constant, that is, $f[X]\subset\{0\}$ or $f[X]\subset\{1\}$.
\end{define}

\begin{theorem} \label{th:conneq}
Let $(X,\varrho)$ be a metric space and $\Xi$ be generated by $(X,\varrho)$. Then $(X,\varrho)$ is connected if and only if $(X,\Xi)$ is connected.
\end{theorem}
\begin{proof}
Let $(X,\Xi)$ be connected. For contradiction let us suppose $U,V$ are subsets of $X$, both open non-empty, $X=U\cup V$ and $U\cap V=\emptyset$. We define $f:X\to D$ as follows
\[
	f(x):= \left\{ \begin{array}{ll}1, & \mathrm{if}\ x\in U; \\ 0, & \mathrm{if}\ x\in V. \end{array} \right.
\]
Then $f$ is continuous and $f[X]=D$. That is a contradiction with $(X,\Xi)$ being connected.

Let now $(X,\varrho)$ be connected. For contradiction let us suppose $f:X\to D$ is a continuous mapping such that $f[X]=D$. Then preimages of one-point (open) sets $U:=f^{-1}[\{0\}]$, $V:=f^{-1}[\{1\}]$ are open in $(X,\varrho)$. Moreover $X=U\cup V$, $U\cap V=\emptyset$ and $U,V$ are non-empty because $\{0,1\}\subset f[X]$. Which is a contradiction with $(X,\varrho)$ being connected.
\end{proof}
\section{Separable spaces}

Again we have a topology so we could use the closure operator and define separability analogically with Definition \ref{def:allprp}. In metric spaces we obtain an equivalent characterisation as it is showed afterwards.

\begin{define}\label{def:sep}
Let $\Xi$ be a \fcs\ on set $X$. We say $\Xi$ is \emph{separable} if there exists a countable subset $A\subset X$ such that
\[
	\forall x\in X\ \exists\{x_n\}\subset A: (\{x_n\},x)\in \Xi.
\]
\end{define}

\begin{corollary} \label{th:conneq}
Let $(X,\varrho)$ be a metric space and $\Xi$ be generated by $(X,\varrho)$. Then $(X,\varrho)$ is separable if and only if $\Xi$ is separable.
\end{corollary}
\begin{proof}
We use Theorem \ref{th:mpseq}\emph{(vi)}.
\end{proof}

It is known that a compact metric space is always separable. The following example shows that this implication does not hold for sequential spaces.

\begin{example}\label{ex:madnw1}
Let $X:=\bigcup_{\beta<\omega_1}X_{\beta}$ where $X_\beta$ are defined by a transfinite induction:
\[
	X_1:=\mathbb{N}, \qquad
	X_\alpha:=\mathrm{MAD}(X_{\alpha-1}), \quad
	X_\gamma:=\mathrm{MAD}(\mathcal{K_\gamma}), \quad
	%X_\alpha:=\mathrm{MAD}(\bigcup_{\beta<\alpha}X_{\beta}),
\]
where $\alpha$ is a successor ordinal, $\gamma$ is a limit ordinal and $\mathcal{K_\gamma}$ denotes the family of sequences that have a finite intersection with $X_\beta$ for every $\beta<\gamma$. We say $X_\beta$ is a \emph{level} for every $\beta$. On $X$ we take a \fcs\ defined analogically to Example \ref{ex:madn}.

Then $X$ is compact. Really, let $\{x_n\}$ be a sequence in $X$. It follows that either $\{x_n\}$ contains an infinite number of members in some level or it has a finite intersection with every level. In the first case we can easily find a convergent subsequence. In the second case, since $\{x_n\}$ has countably many members there exists an $\alpha_0<\omega_1$ such that $\{x_n\}\subset\bigcup_{\beta<\alpha_0}X_\beta$ so it also has a convergent subsequence.

Let $A\subset X$ be a countable set. Then again there exists an $\alpha_0<\omega_1$ such that $A\subset\bigcup_{\beta<\alpha_0}X_\beta$. For every $x\in X_{\alpha_0+1}$ no subsequence $\{x_n\}\subset A$ converges to $x$. That is, $X$ is not separable.
\end{example}

% Ukázka použití některých konstrukcí LateXu (odkomentujte, chcete-li)
%\include{sample}

\include{conclusion}

%%% Seznam použité literatury
%%% Seznam použité literatury je zpracován podle platných standardů. Povinnou citační
%%% normou pro bakalářskou práci je ISO 690. Jména časopisů lze uvádět zkráceně, ale jen
%%% v kodifikované podobě. Všechny použité zdroje a prameny musí být řádně citovány.

\def\bibname{Bibliography}
\begin{thebibliography}{99}
\addcontentsline{toc}{chapter}{\bibname}

\bibitem{cech66}
  {\sc {\v{C}}ech,} Eduard.
  \emph{Bodov{\'e} mno{\v{z}}iny}.
  Vydání třetí nezměněné.
  Praha: Academia, 1974.
  284 p.

\bibitem{copson88}
  {\sc Copson,} Edward Thomas.
  \emph{Metric Spaces}.
  First paperback edition.
  Cambridge: Cambridge University Press, 1988.
  143 p.
  Cambridge Tracts in Mathematics; No. 57.
  ISBN 0-521-35732-2.
  
  
\bibitem{csaszar78} 
  {\sc Császár,} Ákos. 
  \emph{General topology}.
  Budapest: Akadémiai Kiadó, 1978.
  488 p.
  ISBN 96-305-0970-9.


%\bibitem{efremovic53}  
%  {\sc Efremovič,} V. A.; {\sc Švarc}, A. S.
%  \emph{A new definition of uniform spaces.
%  Metrization of proximity spaces}. (Russian)
%  Doklady Akad. Nauk SSSR 89
%  (1953), 393–396.

\bibitem{efremovic53}  
\begin{otherlanguage}{russian}
  {\sc Ефремович,} В. А.; {\sc Шварц}, А. С.
  \emph{Новое определение равномерных пространств. Метризация пространств близостию}.
  Доклады Академии Наук СССР 89
  (1953), 393–396.
\end{otherlanguage}

\bibitem{engelking89}
  {\sc Engelking,} Ryszard.
  \emph{General Topology}.
  Revised and completed edition.
  Berlin: Heldermann Verlag, 1989.
  529 p.
  Sigma Series in Pure Mathematics; Vol. 6.
  ISBN 3-8538-006-4.
  
  
\bibitem{franklin65}
  {\sc Franklin,} Stanley Phillip.
  \emph{Spaces in which sequences suffice}.
  Fund. Math., 57 (1967),
  101-115.

  
\end{thebibliography}


%%% Tabulky v bakalářské práci, existují-li.
%\chapwithtoc{List of Tables}

%%% Použité zkratky v bakalářské práci, existují-li, včetně jejich vysvětlení.
\chapwithtoc{List of Abbreviations}

%%% Přílohy k bakalářské práci, existují-li (různé dodatky jako výpisy programů,
%%% diagramy apod.). Každá příloha musí být alespoň jednou odkazována z vlastního
%%% textu práce. Přílohy se číslují.
%\chapwithtoc{Attachments}

\openright
\end{document}
