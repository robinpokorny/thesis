\chapter*{Preface}
\addcontentsline{toc}{chapter}{Preface}

In the theory of metric spaces various properties are studied. These properties usually have numerous characterisations. We shall examine whether selected ones can be described using convergent sequences.

In the first chapter we remind the definition of a metric space and some of its interesting properties without using sequences. Stated theorems show equivalent conditions or basic behaviour. 

The second chapter introduces two abstract spaces: sequential and uniformly sequential spaces. Both of them provide a generalization of certain properties of sequences we examined in the first chapter.  We show their relations to metric spaces and to each other.

In the third chapter properties are divided into three parts. Countinuous mappings, topology, compactness, connectedness and separability are those which can be characterised using sequential spaces only. We also examine separability behaves slightly differently althrought equivalently in metric spaces. Total boundedness and completeness need more than this, we show that can be characterised using uniformly sequential spaces. The last part concerns properties which cannot be characterised using either of these structures, we show boundedness is one of them.